\documentclass[10pt, onecolumn, letterpaper, draftclsnofoot]{IEEEtran}

\usepackage{geometry}
\geometry{margin=0.75in}

\usepackage[utf8]{inputenc}
\fontfamily{cmss}

% used for bash script
\usepackage{listings}
\usepackage{xcolor}

\usepackage{setspace} \singlespacing

\title{Assignment 1}
\author{Miles Davies, Sonica Gupta, Ryan Sisco \\ Group 29 \\ CS 444: Spring 2018}

\begin{document}

\maketitle
\begin{abstract}
\noindent This assignment write up shows the steps we took to set up the yoctoproject in
our group's directory and then how we built and ran the kernal with gdb. After
we got the kernal set up, we were able to look closely at the commands that we 
used to set up the QEMU environment and more importantly the flags that were 
used, which in turn influenced what we could do. As a whole, this project allowed 
our group to actually gain insight into how gdb can be utilized to debug a kernal 
in a Linux environment.
\end{abstract}

\newpage

\tableofcontents
\newpage 

\section{Command Log}
\lstset{
commentstyle=\color{green},
keepspaces=true,
keywordstyle=\color{blue},
%numbers=left,
%numbersep=5pt,
%numberstyle=\tiny\color{gray},
rulecolor=\color{black},
stringstyle=\color{orange},
%title=\lstname
}

\lstinputlisting[language=bash]{commandScript}

\section{QEMU Flag Explanation}
\begin{itemize}
%remember to cite....
\item{-gdb tcp::5529 -s \ -kernal bzImage-qemux86.ext4 \--append "root=/dev/vda rw console=ttyS0 debug"} \\
This series of flags allows QEMU to connect and work with GDB. So, the first part 
specifies the port (for us: 5500+29 = 5529). This is important because then when we
get back to the place we run GDB, we would need to specify this port to connect to. 
Then the next part boots up and runs the kernal \cite{DOC:1}.
\item{-nographic} \\
This flag deals with the display. The -nograpic flag allows QEMU to be displayed as
a command line application instead of in a Window with graphics, which allows us to 
use this to debug a kernal in Linux \cite{DOC:1}. 
\item{-drive file=core-image--lsb-sdk-qemux86.ext4,if=virtio} \\
This flag is how you would specify the image to use with the drive. So, in this case
it is saying core-image--lesb-sdk-qemux86.ext4 is the file for the disk image of the 
drive. Then the if statement says that the drive is connected with a virtio type \cite{MAN:2}.
\item{-enable-kvm} \\
This allows it to be compiled with kvm (Kernal based Virtual Machine). Without this
flag, QEMU would run in user-space \cite{MAN:2}. 
\item{-net none} \\
The -net is used to create a NIC (Network Interface Card). If this flag was not 
specified, a NIC would have been created. But since it was specified to 'none', no NIC
was created \cite{MAN:2}.
\item{-usb} \\
With this flag USB devices could be connected. So, if the flag had been -use-mouse
that would indicate a mouse connected via usb. But, since no flag followed, that means
nothing connected through usb \cite{DOC:1}. 
\item{-localtime} \\
This flag allows the date to be set accurately \cite{DOC:1}.
\item{--no-reboot} \\ 
When someone types "reboot" into the QEMU space, it will close and instead of starting
back up again, as you would expect with reboot, it would just exit \cite{DOC:1}.
\end{itemize}

\section{Given Questions}
\begin{enumerate}
\item{What do you think the main point of the assignment is?} \\
The main point of this assignment is to get aquainted with booting a kernal onto a VM
and understanding QEMU. The assignment as a whole only needed a few commands to 
actually install yocto and boot the kernal; however, getting to the place to 
understand what the commands mean and what output to look for was tricky, especially 
because there is no direct output. For us, this meant we had to give it a few tries 
before we actually felt comfortable with what we were doing, which is important 
especially for future assignments that would build off of this. \par
Additionally, this assignment helped us understand QEMU. As we started to look at the 
different flags, it helped us break down the very large command, which in turn made 
it more understandable. It was also interesting to look at the QEMU documentation 
because there is so many flags and options available. \par
\item{How did you personally approach the problem? Design decisions, algorithm, etx.} \\
I approached this problem by reading the README.md that yocto provides because there is alot of information about how to build and set up the kernal. It not only shows the 
commands needed but what each command is doing, which makes it useful to understand 
what is happending. \par
\item{How did you ensure your solution was correct? Testing details, for instance.} \\
I did not know if the we had been building the kernal correctly until I got to the last
stages of the assignment. At that point when you connect the kernal and the gdb portion 
together through the port, you can see it running simultaneously and interacting with 
each other over different terminal sessions. For instance, when I typed in "reboot" 
into the kernal, gdb told me that a process had stopped, which it did. That is what 
had indicated to me that it was built and functioning correctly. \par
\item{What did you lean?} \\
Going into this assignment, I knew very little to nothing about QEMU and kernal. I 
still do not know how much I know about them both but I now see how GDB could be 
utilized as a debugger for the QEMU environment. Additionally, I learned how the 
flags could be changed deppending on what is necessary. More importantly, I also
learned how to build and run a kernal, which I feel will be utilized and built on in 
comming assignments. \par
\end{enumerate}

\section{Version Control Log} 
Used github as version control system and git log for the log info \par
\begin{table}[h]
\centering
\label{my-label}
\begin{tabular}{|l|l|l|l|}
\hline
        & Name         & Date                     & Message                                            \\ \hline
3135eb9 & Sonica       & Sun Apr 15 21:55:00 2018 & added bibliography + citations                     \\ \hline
5cb1629 & Sonica       & Sun Apr 15 21:08:03 2018 & changed for IEEE styling                           \\ \hline
f127fd1 & Sonica Gupta & Sun Apr 15 20:52:14 2018 & added abstract                                     \\ \hline
89a808c & Sonica Gupta & Sun Apr 15 20:42:10 2018 & added work log                                     \\ \hline
4cfd3d1 & Sonica Gupta & Sun Apr 15 17:17:22 2018 & removed a line                                     \\ \hline
c25282c & Sonica Gupta & Sun Apr 15 17:06:57 2018 & added flag explanation, answers to given questions \\ \hline
5f1f961 & Sonica Gupta & Sun Apr 15 16:07:10 2018 & added command log section                          \\ \hline
21002ed & Sonica Gupta & Sun Apr 15 15:25:49 2018 & finished formatting skeleton, reflects req.        \\ \hline
c6e1241 & Sonica Gupta & Sun Apr 15 01:35:07 2018 & removed files                                      \\ \hline
354e69b & Sonica Gupta & Sun Apr 15 01:19:33 2018 & added files and updated .tex skeleton              \\ \hline
14cccbe & Sonica Gupta & Sun Apr 15 00:36:01      & undo file                                          \\ \hline
329ac42 & Sonica Gupta & Sun Apr 15 00:26:15      & set up files                                       \\ \hline
22e227c & Sonica Gupta & Sun Apr 15 00:11:46      & basic skelton for write-up                         \\ \hline
998a3a7 & Sonica Gupta & Sat Apr 14 23:51:25 2018 & first commit                                       \\ \hline
\end{tabular}
\end{table}


\section{Work log}
Our group had met on Friday evening to work on the coding part of the assignment 
together. We had worked through installing the yocto git repository and then building 
the kernal. Once that had been set up, we looked at the QEMU flags to see in detail 
what they were doing and then discussed them so that we would ready to do the write up. 
Then once we had all that settled, we worked on the write up through the weekend,
slowly adding parts to the document \cite{Doc:1}.

\newpage
\bibliography{assign1}
\bibliographystyle{ieeetr}

\end {document}
